Cell-free systems are a promising technology to avoid current legal limitations restricting the usage of genetically modified organisms. There is a need for development of new systems using the cell-free approach, but most attempts have been focused on mimicking normal cell behaviour. This work has as principal aim the generation of a modular cell-free system capable of not only sensing variables present in the environment, such as heavy metals or aromatic compounds, but also analysing them through the use of in vitro logic gates, hereafter called DNAlogic. The generation of memory, together with the use of RNA aptamers as final outputs, will constitute a solution for problems commonly observed in cell free system applications. Also, the use of CRISPR/Cpf1 in a cutting and ligation processing machinery will generate a new synthetic and adaptable trigger, which would make possible the generation of more in vitro DNA logic systems. 