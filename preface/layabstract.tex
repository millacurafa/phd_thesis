In an era where Genetically Modified Organisms (GMOs) are a mainstream topic within society, a necessity for different approaches without the usage of living organisms is needed. One of the most promising technologies is the use of Cell-free systems which doesn't imply a living prospect applicable to current legal or moral society restrictions.

During Cell-free preparation the cell wall is ripped off, the insides collected, and the cell catalyst used to produce new kinds of molecules and biological processes without the evolutionary constraints of using intact living cells.

In this work a novel cell-free system approach that allows the detection of toxic pollutants, such as heavy metals, and also able to analyse mixed samples as real environmental samples commonly carrying more than just one contaminant is presented. 

Simple biological molecules such as RNA are used in our system to accelerate the detection process, allowing us to use different approaches rather than imitating ordinary cell conduct, which is the most common scientific rationale to approach these kind of problems.