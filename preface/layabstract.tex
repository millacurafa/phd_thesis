In an era where Genetically Modified Organisms (GMOs) are a mainstream topic within society,
there is a need for different approaches without the usage of living organisms. One of the 
most promising technologies is the use of Cell-free systems, which don't imply a living prospect
applicable to current legal or moral society restrictions.

During Cell-free preparation the cell wall is ripped off, the insides collected, and the cell
catalysts used to produce new kinds of molecules and biological processes without the evolutionary
constraints of using intact living cells.

In this work novel cell-based and cell-free system approach were developed. These allow the detection
of pollutants, such as heavy metals or antibiotics, which are used as input signals and later analysed
in binary using logic gates, just as it happen in computers. 

Simple biological molecules, such as RNA, are used to accelerate the detection process, allowing us to
avoid imitating ordinary cell conduct, which is the most common scientific rationale to solve these kind 
of problems. RNA aptamers are used as outputs allowing the generation of fast, reliable and simple-to-design
detection units. These cell-free reactions and their possible applications become a promising new tool for
fast and simple bench-to-market genetic circuit and biosensor applications.

Working with computer-like behaviour, a system that decomposes large operations into smaller ones is here explained.
This system allows faster analysis by processing information in a collection of small subcircuits instead of a large
one, just like it happens in a Graphical Processing  unit (GPU), also known as parallel computing. This was further 
demonstrated by creating a calculator-like display that shows a numeric result, from 0 to 7, when the proper 3 bit 
binary input is introduced into the system. This parallel approach facilitates the analysis avoiding the need for 
complex genetic engineering, solving some legal and ethical implications.

Additionally, a versatile cell-free system based on the extremely tolerant bacteria \textit{Cupriavidus  metallidurans} CH34
is shown here. This is able of not only sensing environmental variables, such as heavy metals, but also synthesize proteins
and produce bioplastic. This novel cell-free chassis comes after the discovery of the unstable genome that this bacteria carries,
which is also explained in this work, offering novel possibilities of development considering the cell free approach.